\documentclass[sigplan,review,11pt,nonacm,natbib=false]{acmart}
\settopmatter{printfolios=false,printccs=false,printacmref=false}
\usepackage[maxnames=1,minnames=1,maxbibnames=100,natbib=true,citestyle=authoryear,bibstyle=authoryear,doi=false,url=false,isbn=false,isbn=false]{biblatex}
\usepackage[utf8]{inputenc}
\usepackage{listings}
\usepackage{graphicx}
\usepackage{xcolor}
\usepackage{ffcode}
\usepackage{amsmath}

\newtheorem{definition}{Definition}[section]

\graphicspath{ {./images/} }


\title{Elimination Of Jump Objects}
\author{author}

\begin{document}

\maketitle


\keywords{Object-Oriented Programming, software engineering, algorithms, computer science}

\section{Introduction}
The problem is known both computer science and soft-
ware engineering communities for a few decades and formulated by several scientists.

Many of them have offered solutions to this problem. (Look at Related Work)

The presence of jump statements in the source code makes it impossible to use the functional method of analysis.

There are an "object-flow" languages where we can control flow through an objects. An example of such language is EO (Elementary Objects language). This is a strictly object-oriented programming language in which any action is described by creating a new object, or decorating an existing one.

We are trying to answer the following research question: "Is it possible for any program, written on "object-flow" language like EO and containing GOTO objects to find a semantically equivalent program, but without using them?"

\section{Related work}
To solve the problem of eliminating Jump statements in other programming languages, the following works were published:
Morris suggested to use regular expressions for replacing GOTO with IF-THEN-ELSE constructs;
Ramshow proposed an algorithm for Pascal programs;
Ceccato suggested how to eliminate GOTO during migration of legacy code to Java.

To our knowledge, there is still no method which enables automatic elimination of jump objects in programs written in such an object-flow languages.

\section{Method}
In this article, we will describe an algorithm that will convert an EO program in such a way that the output returns a semantically equivalent program, but without using the GOTO object.

Some objects in EO programs may need to be platform specific and can’t be composed from other existing objects—they are called \emph{atoms}. 
So all jump statements are implemented as a
single \emph{atom}-object GOTO. The implementation of goto-object in EO implies two types of jumps - goto forward and goto backward. So we should consider transformations for both cases.

Since we consider a solution to the problem in strictly object-oriented programming languages, which means that the whole program is a set of objects where some objects can contain others.
It should be noted that each object in the program has its own level of \emph{nesting} relative to other objects.

\begin{definition} The \emph{nesting} of the current object means the number of objects that contain the current object directly, or that contain objects that contain the current object.
\end{definition}

The solution we propose contains two types of transformations - forward jump and backward jump.
The algorithm itself consists in applying these transformations to replace both types of jumps.

\vspace{5mm}

\emph{Designations:}

\begin{itemize}
\item $\sigma$ - Denotes either an object, a chain of nested objects or the absence of an object.

\item $\delta$ - Denotes a not-empty object or a chain of nested objects that return a not-NULL value when accessed.

\item $\mu$ - Denotes an object or a chain of nested objects that return a boolean value (TRUE/FALSE) when executed.
\end{itemize}

\newcommand\sgm[1]{$\sigma_#1$}
\newcommand\dlt[1]{$\delta_#1$}
\newcommand\m[1]{$\mu_#1$}

We will also look at our transformations in the form of $\varphi$-calculus, on which EO is based. The proposed $\varphi$-calculus represents an object model through data and objects, while operations with them are possible through abstraction, application, and decoration.
The calculus introduces a formal apparatus for manipulations with objects.

$\varphi$-calculus uses an alternative “arrow notation” to denote an object in a more compact way, where free attributes of an abstract object stay in the parentheses on the left side of the mapping symbol $\mapsto$ and pairs, which represent bound attributes, stay on the right side, in double-square brackets.


\subsection{Forward jump}
Goto forward in EO is implemented as:

\begin{ffcode}
+import org.eolang.gray.goto
+import org.eolang.io.stdout

goto
  [g]
    |\sgm{1}| > @
      g.forward |\dlt{1}|
      stdout "Will never be printed"
\end{ffcode}

It consists of the goto atom-object, the abstract object \emph{g}, and the g.forward attribute. Several abstract objects such as \emph{g} can be declared inside of goto-object, and the attribute \emph{.forward} can be called to each of them.

\subsubsection{Simple Forward}
Consider the simplest example of using goto-forward:

\begin{ffcode}
goto
  [g]
    g.forward |\dlt{1}| > @
\end{ffcode}

In terms of $\varphi$-calculus, this code will appear as:

\begin{equation}
\begin{split}
& \ff{goto} ( \alpha_1 \mapsto \llbracket \ff{g} \mapsto \varnothing, \varphi \mapsto \ff{g}.\ff{forward} (\delta_1) \rrbracket ) \\
\end{split}
\end{equation}

Note that in this example, g.forward is simply an attribute of the abstract object \emph{g} and is executed as soon as the queue reaches it.
Our transformation only considers the jumps that are made when a condition for it is fulfilled. In cases where there is unconditional jump, first of all we transform it into conditional jump by adding an additional if-statement - if (TRUE).
This will not violate the program logic, because without the added if-statement, the jump would always be executed, and statements coming after this jump with a nesting not less that the nesting of the abstract object \emph{g} would never be executed:

\begin{equation}
\begin{split}
& \ff{goto}(\alpha_1 \mapsto \llbracket \ff{g} \mapsto \varnothing, \\
& \quad \varphi \mapsto \ff{if}.( TRUE, \ff{g}.\ff{forward}(\delta_1), TRUE ) \rrbracket) 
\end{split}
\end{equation}

Same example on EO:

\begin{ffcode}
goto
  [g]
    if. > @
      TRUE
      g.forward |\dlt{1}|
      TRUE
\end{ffcode}


\subsubsection{Goto forward wrapped in an object}

After the conversion described in section 3.1.1, the simplest call of forward jump will be presented as:

\begin{equation}
\begin{split}
& \ff{goto} ( \alpha_1 \mapsto \llbracket \ff{g} \mapsto \varnothing, \\
& \quad \varphi \mapsto \ff{if}.(\mu_1, \ff{g}.\ff{forward}(\delta_1), \delta_2) \rrbracket) \\
\end{split}
\end{equation}

The transformation under consideration involves the following steps:

\begin{equation}
\begin{split}
& \ff{goto} ( \alpha_1 \mapsto \llbracket \ff{g} \mapsto \varnothing, \\
& \quad \varphi \mapsto \ff{if}.(\mu_1.\ff{not}, \ff{g}.\ff{forward}(\delta_1), \delta_2) \rrbracket) \\
\end{split}
\end{equation}

\begin{equation}
\begin{split}
& \ff{goto} ( \alpha_1 \mapsto \llbracket \ff{g} \mapsto \varnothing, \\
& \quad \varphi \mapsto \ff{if}.(\mu_1.\ff{not}, \delta_2, \delta_1) \rrbracket) \\
\end{split}
\end{equation}

\begin{equation}
\begin{split}
& \ff{if.}(\mu_1.not, \delta_2, \delta_1) \\
\end{split}
\end{equation}

\begin{theorem}
Expression 3 is semantically equivalent to expression 6:
\end{theorem}

\begin{equation} \nonumber
\begin{split}
& \ff{goto} ( \alpha_1 \mapsto \llbracket \ff{g} \mapsto \varnothing, \\
& \quad \varphi \mapsto \ff{if}.(\mu_1, \ff{g}.\ff{forward}(\delta_1), \delta_2) \rrbracket) \\
\end{split}
\end{equation}

\hline

\begin{equation} \nonumber
\begin{split}
& \ff{if.}(\mu_1.not, \delta_2, \delta_1) \\
\end{split}
\end{equation}

To prove that the transformation does not violate program logic, let's consider it with the following abstract case of calling goto forward.

Consider the following code:

\begin{ffcode}
goto
  [g]
    |\sgm{1}| > @ |$\label{ln:ret0}$|
      |\sgm{2}|
      if.
        |\m{1}|
        g.forward |\dlt{1}|
        |\dlt{2}|
      |\sgm{3}| |$\label{ln:ret1}$|
\end{ffcode}

Let's denote lines \ref{ln:ret0}-\ref{ln:ret1} as X.

It is not difficult to notice that statements which goto-forward "jumps through" will be executed if the condition in the if-statement is not equal to TRUE.
Otherwise statements following the goto-object itself are executed.
So we can swap statements and invert the condition into if-statement to get rid of the \emph{g.forward} attribute without violating program logic.

Also, after replacing the jump, we must create an additional \emph{flag} object, which will be changed if the jump condition is met, i.e. when in the new if-statement, the 'ELSE' branch will be executed - value of flag will change. This is required so that all objects following the if-statement that are attributes of the abstract object g are not executed.
So, by wrapping each object following the jump condition and whose nesting is at least that of the g object in an if-statement with a flag value check condition, we make sure that these objects will not be executed when the jump condition is met.

By swapping the statements that follow the if-statement body and after the jump itself, as described above, we get the following replacement for this example: the \m{1} object must be inverted with (\m{1}).not; in 'THEN' brunch of if-statement places \dlt{2}; in the 'ELSE' brunch creates new object seq, consisting of two elements - \dlt{1} and flag's value change, because the jump condition is satisfied.

Thus, after transformation, the program converts to:
\begin{ffcode}
memory > flag
flag.write 0
|\sgm{1}||$\label{ln:ret2}$|
  |\sgm{2}|
  if.
    |\m{1}|.not
    |\dlt{2}|
    seq
      |\dlt{1}|
      flag.write 1
  if.
    (eq. (flag 1)).not
    |\sgm{3}|
    TRUE|$\label{ln:ret3}$|
\end{ffcode}

Let's denote lines \ref{ln:ret2}-\ref{ln:ret3} as X'.

Now let's look at an example program with nested objects containing goto forward:
\begin{ffcode}
goto
  [g]
    |\sgm{0}| > @
      X
      |\sgm{4}|
\end{ffcode}

Applying the transformation, we get:
\begin{ffcode}
memory > flag
flag.write 0
|\sgm{0}|
  X'
  if.
    (eq. (flag 1)).not
    |\sgm{4}|
    TRUE
\end{ffcode}


\subsubsection{Goto forward inside the while-loop}
In order to realize the functionality of cycles in EO, generators are provided. A generator is a routine that can be used to control the iteration behaviour of a loop.

The language currently provides a while generator, on which other generators are implemented.

The simplest way to call goto forward inside a while-loop is represented as:

\begin{equation}
\begin{split}
& \ff{goto} ( \alpha_1 \mapsto \llbracket \ff{g} \mapsto \varnothing, \\
& \quad \varphi \mapsto \ff{while}.(\mu_0, \\
& \quad \quad \ff{if}.(\mu_1, \ff{g}.\ff{forward}(\delta_1), \delta_2) \\
& \quad ) \\
& \rrbracket) \\
\end{split}
\end{equation}

The transformation under consideration involves the following steps:

%%\begin{equation}
%%\begin{split}
%%& \ff{memory}(flag), \\
%%& flag.\ff{write}(0), \\
%%& \ff{goto} ( \alpha_1 \mapsto \llbracket \ff{g} %%\mapsto \varnothing, \\
%%& \quad \varphi \mapsto \ff{while}.(\mu_0, \\
%%& \quad \ff{if}.(\mu_1, %%\ff{g}.\ff{forward}(\delta_1), \delta_2) \\
%%& \quad )\rrbracket) \\
%%\end{split}
%%\end{equation}

\begin{equation}
\begin{split}
& \ff{memory}(flag), \\
& flag.\ff{write}(0), \\
& \ff{goto} ( \alpha_1 \mapsto \llbracket \ff{g} \mapsto \varnothing, \\
& \quad \varphi \mapsto \ff{while}.(\ff{and}.( \\
& \quad \quad \mu_0, \ff{eq}.(flag  1).\ff{not}), \\
& \quad \quad \ff{if}.(\mu_1,  \ff{g}.\ff{forward}(\delta_1), \delta_2) \\
& \quad ) \\
& \rrbracket) \\
\end{split}
\end{equation}

\begin{equation}
\begin{split}
& \ff{memory}(flag), \\
& flag.\ff{write}(0), \\
& \ff{goto} ( \alpha_1 \mapsto \llbracket \ff{g} \mapsto \varnothing, \\
& \quad \varphi \mapsto \ff{while}.(\ff{and}.( \\
& \quad \quad \mu_0, \ff{eq}.(flag  1).\ff{not}), \\
& \quad \quad \ff{if}.(\mu_1.\ff{not}, \delta_2, \\
& \quad \quad \quad \ff{seq}( \delta_1, flag.\ff{write}(0) )\\
& \quad \quad ) \\
& \quad ) \\
& \rrbracket) \\
\end{split}
\end{equation}

\begin{equation}
\begin{split}
& \ff{memory}(flag), \\
& flag.\ff{write}(0), \\
& \ff{while}.(\ff{and}.( \\
& \quad \mu_0, \ff{eq}.(flag  1).\ff{not}), \\
& \quad \ff{if}.(\mu_1.\ff{not}, \delta_2, \\
& \quad \quad \ff{seq}( \delta_1, flag.\ff{write}(0) )\\
& \quad ) \\
& ) \\
\end{split}
\end{equation}

\begin{theorem}
Expression 7 is semantically equivalent to expression 10:
\end{theorem}

\begin{equation} \nonumber
\begin{split}
& \ff{goto} ( \alpha_1 \mapsto \llbracket \ff{g} \mapsto \varnothing, \\
& \quad \varphi \mapsto \ff{while}.(\mu_0, \\
& \quad \quad \ff{if}.(\mu_1, \ff{g}.\ff{forward}(\delta_1), \delta_2) \\
& \quad ) \\
& \rrbracket) \\
\end{split}
\end{equation}

\hline

\begin{equation} \nonumber
\begin{split}
& \ff{memory}(flag), \\
& flag.\ff{write}(0), \\
& \ff{while}.(\ff{and}.( \\
& \quad \mu_0, \ff{eq}.(flag  1).\ff{not}), \\
& \quad \ff{if}.(\mu_1.\ff{not}, \delta_2, \\
& \quad \quad \ff{seq}( \delta_1, flag.\ff{write}(0) \\
& \quad )\\
& ) \\
\end{split}
\end{equation}

To prove that the transformation does not violate program logic, let's consider it with the following abstract case of calling goto forward.

Consider the following code on EO:

\begin{ffcode}
goto
  [g]
    |\sgm{0}| > @ |$\label{ln:ret4}$|
      while.
        |\m{0}|
        |\sgm{1}|
          |\sgm{2}|
          if.
            |\m{1}|
            g.forward |\dlt{1}|
            |\dlt{2}|
          |\sgm{3}||$\label{ln:ret5}$|
\end{ffcode}

Let's denote lines \ref{ln:ret4}-\ref{ln:ret5} as Y.

In this case, goto-forward is used to exit the body of the while loop. However, every iteration of a while loop checks its own condition if it is true. 

Since the jump is executed once the jump condition is true, the while loop will be executed as long as the jump condition is false. We can deduce that the condition for the jump is tested at every iteration of the loop, exactly the same as the condition in the while body itself.
So by combining the jump condition and the while loop condition do not violate the program logic. But the point is, once the jump condition is true, the next iteration of the while loop will fail and the loop will end.
So we only care about the first triggering of the jump condition for goto-forward. To do this, we can create an auxiliary object \emph{flag} that will be TRUE as soon as the jump condition is met.
Otherwise, all statements that came after the jump in the while loop will be executed.
So the combined condition of the while loop will look like: while (!flag AND condOfWhile).

Thus, the transformation is:

\begin{ffcode}
memory > flag
flag.write 0
|\sgm{0}||$\label{ln:ret6}$|
  while.
    and.
      (eq. (flag 1)).not
      |\m{0}|
    |\sgm{1}|
      |\sgm{2}|
      if.
        |\m{1}|.not
        |\dlt{2}|
        seq
          |\dlt{1}|
          flag.write 1
      if.
        (eq. (flag 1)).not
        |\sgm{3}|
        TRUE|$\label{ln:ret7}$|
\end{ffcode}

Let's denote lines \ref{ln:ret6}-\ref{ln:ret7} as Y'.

We create an additional \emph{flag} object that will be equal to 1 when the jump condition is triggered and add this condition to the loop body; modify if-statement as described in Section 3.1.2, except that in the 'ELSE' branch we are changing the flag value because the jump condition has been met and the next iteration of the loop should not be executed.

It is worth noting that if goto-forward represent several nested while-loops, then in each of them should be added to the condition of the loop check whether the flag is equal to 1, as in the inner while, from where the jump is made. So, every while that has a nesting less than the original nesting and more than the nesting of the abstract object \emph{g} will end without performing the next iteration.

For example, a program with nested while-loops may look like this:
\begin{ffcode}
goto
  [g]
    while. > @
      |\m{9}|
      |\sgm{9}|
        Y
        |\sgm{4}|
\end{ffcode}

After transformation it will look like:
\begin{ffcode}
memory > flag
flag.write 0
while.
  and.
    (eq. (flag 1)).not
    |\m{9}|
  |\sgm{9}|
    Y'
    if.
      (eq. (flag 1)).not
      |\sgm{4}|
      TRUE
\end{ffcode}






\subsection{Backward jump}
Goto backward in EO is implemented as:

\begin{ffcode}
+import org.eolang.gray.goto
+import org.eolang.io.stdout

goto
  [g]
    |\sgm{1}| > @
      stdout "Will be printed forever"
      g.backward
\end{ffcode}

Same as with goto forward, there is the goto \emph{atom}-object, the abstract object \emph{g}, and the g.backward attribute. Several abstract objects such as \emph{g} can be declared inside of goto-object, and the attribute \emph{.backward} can be called to each of them.

\subsubsection{Simple Backward} The simplest example of using goto-backward:

\begin{ffcode}
goto
  [g]
    g.backward > @
\end{ffcode}

which in terms of $\varphi$-calculus appears as:

\begin{equation}
\begin{split}
& \ff{goto}(\alpha_1 \mapsto \llbracket \ff{g} \mapsto \varnothing, \varphi \mapsto \ff{g}.\ff{backward} \rrbracket ) \\
\end{split}
\end{equation}

This is identical to that discussed in Section 3.1.1. In cases where the jump is performed without condition, you need to add an additional if-statement, similar to the example in Section 3.1.1:

\begin{equation}
\begin{split}
& \ff{goto}(\alpha_1 \mapsto \llbracket \ff{g} \mapsto \varnothing, \\
& \quad \varphi \mapsto \ff{if}.( TRUE, \ff{g}.\ff{backward}, TRUE ) \rrbracket) \\
\end{split}
\end{equation}

Same code on EO:

\begin{ffcode}
goto
  [g]
    if. > @
      TRUE
      g.backward
      TRUE
\end{ffcode}

\subsubsection{Goto backward wrapped in an object}
After the conversion described in Section 3.2.1, the simplest call of backward jump will be presented as:

\begin{equation}
\begin{split}
& \ff{goto}(\alpha_1 \mapsto \llbracket \ff{g} \mapsto \varnothing, \\
& \quad \varphi \mapsto \ff{if}.( \mu_1, \ff{g}.\ff{backward}, \delta_1 ) \rrbracket) \\
\end{split}
\end{equation}

The transformation under consideration involves the following steps:

\begin{equation}
\begin{split}
& \ff{memory}(flag), \\
& flag.\ff{write}(0), \\
& \ff{goto}(\alpha_1 \mapsto \llbracket \ff{g} \mapsto \varnothing, \\
& \quad \varphi \mapsto \ff{while}.(\ff{eq}.(flag 0), \\
& \quad \quad \ff{if}.( \mu_1, \ff{g}.\ff{backward}, \delta_1 ) \\
& \quad ) \\
& \rrbracket) \\
\end{split}
\end{equation}

\begin{equation}
\begin{split}
& \ff{memory}(flag), \\
& flag.\ff{write}(0), \\
& \ff{goto}(\alpha_1 \mapsto \llbracket \ff{g} \mapsto \varnothing, \\
& \quad \varphi \mapsto \ff{while}.(\ff{eq}.(flag 0), \\
& \quad \quad \ff{seq}(flag.\ff{write}(1), \\
& \quad \quad  \quad \ff{if}.( \mu_1.\ff{not}, \delta_1, flag.\ff{write}(0) ) \\
& \quad \quad ) \\
& \quad ) \\
& \rrbracket) \\
\end{split}
\end{equation}

\begin{equation}
\begin{split}
& \ff{memory}(flag), \\
& flag.\ff{write}(0), \\
& \ff{while}.(\ff{eq}.(flag 0), \\
& \quad \ff{seq}(flag.\ff{write}(1), \\
& \quad \quad \ff{if}.( \mu_1.\ff{not}, \delta_1, flag.\ff{write}(0) ) \\
& \quad ) \\
& ) \\
\end{split}
\end{equation}

\begin{theorem}
Expression 13 is semantically equivalent to expression 16:
\end{theorem}

\begin{equation} \nonumber
\begin{split}
& \ff{goto}(\alpha_1 \mapsto \llbracket \ff{g} \mapsto \varnothing, \\
& \quad \varphi \mapsto \ff{if}.( \mu_1, \ff{g}.\ff{backward}, \delta_1 ) \rrbracket) \\
\end{split}
\end{equation}

\hline

\begin{equation} \nonumber
\begin{split}
& \ff{memory}(flag), \\
& flag.\ff{write}(0), \\
& \ff{while}.(\ff{eq}.(flag 0), \\
& \quad \ff{seq}(flag.\ff{write}(1), \\
& \quad \quad \ff{if}.( \mu_1.\ff{not}, \delta_1, flag.\ff{write}(0) ) \\
& \quad ) \\
& ) \\
\end{split}
\end{equation}

To prove that the transformation does not violate program logic, let's consider it with the following abstract case of calling goto backward.

Consider the following code on EO:

\begin{ffcode}
goto
  [g]
    |\sgm{1}||$\label{ln:ret8}$|
      |\sgm{2}|
      if.
        |\m{1}|
        g.backward
        |\dlt{1}|
      |\sgm{3}||$\label{ln:ret9}$|
\end{ffcode}

Let's denote lines \ref{ln:ret8}-\ref{ln:ret9} as W.

Note that goto-backward works the same as do-while construction in other languages: the first iteration of the new loop is always executed, and the next only if the condition is TRUE.
So, all objects that occurs between the declaration of an abstract object \emph{g} and the "jump" itself should be executed before the new while loop once, and all subsequent times were executed depending on the condition of the loop.

Since there is no direct implementation of the do-while construct in EO, but there is an implementation of the while generator, we can resort to a little trick.

To create a do-while emission, we can create a flag globally that will be a condition for our outer loop, whose execution condition will depend on this value. The initial value of this flag will be such that the condition for the execution of the added while loop is fulfilled. Then, in the body of the loop, we change the value of the flag to the opposite. When the jump condition is met, we again change the value of the flag so that as soon as the iteration of the loop is completed, the next iteration will begin, since the loop execution condition will be met.

In order to ensure that when the jump condition is met, all other objects in the body of the newly added while loop are not executed, we wrap them in an additional if-statement, in which we will check whether the jump condition was previously executed.

So, after transformation, the program converts to:
\begin{ffcode}

memory > flag
flag.write 0
while.
  eq. (flag 0)
  seq
    flag.write 1
    |\sgm{1}||$\label{ln:ret10}$|
      |\sgm{2}|
      if.
        (|\m{1}|).not
        |\dlt{1}|
        flag.write 0
      if.
        (eq. (flag 0)).not
        |\sgm{3}|
        TRUE|$\label{ln:ret11}$|
    
\end{ffcode}

Let's denote lines \ref{ln:ret10}-\ref{ln:ret11} as W'.

Let's look at an example of a chain of nested objects containing a goto backward:

\begin{ffcode}

goto
  [g]
    |\sgm{0}|
      W
      |\sgm{4}|

\end{ffcode}

Applying transformation, we get:
\begin{ffcode}

memory > flag
flag.write 0
while.
  eq. (flag 0)
  seq
    flag.write 1
    |\sgm{0}|
      W'
      if.
        (eq. (flag 0)).not
        |\sgm{4}|
        TRUE

\end{ffcode}


\subsubsection{Goto backward inside the while-loop}
Goto backward also can be called inside a while-loop body.

The simplest way to call goto-backward inside a while-loop is represented as:

\begin{equation}
\begin{split}
& \ff{goto}(\alpha_1 \mapsto \llbracket \ff{g} \mapsto \varnothing, \\
& \quad \varphi \mapsto \ff{while}.(\mu_0, \\
& \quad \quad \ff{if}.(\mu_1, \ff{g}.\ff{backward}, \delta_2) \\
& \quad ) \\
& \rrbracket) \\
\end{split}
\end{equation}

The transformation under consideration involves the following steps:

\begin{equation}
\begin{split}
& \ff{memory}(flag), \\
& flag.\ff{write}(0), \\
& \ff{goto}(\alpha_1 \mapsto \llbracket \ff{g} \mapsto \varnothing, \\
& \quad \varphi \mapsto \ff{while}.(\ff{eq}.(flag 0), \\
& \quad \quad \ff{while}.(\mu_0, \\
& \quad \quad \quad \ff{if}.( \mu_1, \ff{g}.\ff{backward}, \delta_1 ) \\
& \quad \quad ) \\
& \quad ) \\
& \rrbracket) \\
\end{split}
\end{equation}

\begin{equation}
\begin{split}
& \ff{memory}(flag), \\
& flag.\ff{write}(0), \\
& \ff{goto}(\alpha_1 \mapsto \llbracket \ff{g} \mapsto \varnothing, \\
& \quad \varphi \mapsto \ff{while}.(\ff{eq}.(flag 0), \\
& \quad \quad \ff{seq}(flag.\ff{write}(1), \\
& \quad \quad \quad \ff{while}.( \\
& \quad \quad \quad \quad \ff{and}.(\mu_0, \ff{eq}.(flag 0).\ff{not}), \\
& \quad \quad \quad \quad \ff{if}.( \mu_1.\ff{not}, \delta_1, flag.\ff{write}(0) ) \\
& \quad \quad \quad ) \\
& \quad \quad ) \\
& \quad ) \\
& \rrbracket) \\
\end{split}
\end{equation}

\begin{equation}
\begin{split}
& \ff{memory}(flag), \\
& flag.\ff{write}(0), \\
& \ff{while}.(\ff{eq}.(flag 0), \\
& \quad \ff{seq}(flag.\ff{write}(1), \\
& \quad \quad \ff{while}.( \\
& \quad \quad \quad \ff{and}.(\mu_0, \ff{eq}.(flag 0).\ff{not}), \\
& \quad \quad \quad \ff{if}.( \mu_1.\ff{not}, \delta_1, flag.\ff{write}(0) ) \\
& \quad \quad ) \\
& \quad ) \\
&  ) \\
\end{split}
\end{equation}

\begin{theorem}
Expression 17 is semantically equivalent to expression 20:
\end{theorem}

\begin{equation} \nonumber
\begin{split}
& \ff{goto}(\alpha_1 \mapsto \llbracket \ff{g} \mapsto \varnothing, \\
& \quad \varphi \mapsto \ff{while}.(\mu_0, \\
& \quad \quad \ff{if}.(\mu_1, \ff{g}.\ff{backward}, \delta_2) \\
& \quad ) \\
& \rrbracket) \\
\end{split}
\end{equation}

\hline

\begin{equation} \nonumber
\begin{split}
& \ff{memory}(flag), \\
& flag.\ff{write}(0), \\
& \ff{while}.(\ff{eq}.(flag 0), \\
& \quad \ff{seq}(flag.\ff{write}(1), \\
& \quad \quad \ff{while}.( \\
& \quad \quad \quad \ff{and}.(\mu_0, \ff{eq}.(flag 0).\ff{not}), \\
& \quad \quad \quad \ff{if}.( \mu_1.\ff{not}, \delta_1, flag.\ff{write}(0) ) \\
& \quad \quad ) \\
& \quad ) \\
&  ) \\
\end{split}
\end{equation}

To prove that the transformation does not violate program logic, let's consider it with the following abstract case of calling goto backward.

As discussed in section 3.1.3, the behavior of while is different from other objects in the language. Therefore, the transformation will be slightly different from that described in section 3.2.2.

To understand that the logic is not broken, let's look at a special case of calling goto backward inside a while-loop.

Let's consider the following code:

\begin{ffcode}
goto
  [g]
    |\sgm{0}| > @|$\label{ln:ret12}$|
      while.
        |\m{0}|
        |\sgm{1}|
          |\sgm{2}|
          if.
            |\m{1}|
            g.backward
            |\dlt{1}|
          |\sgm{3}||$\label{ln:ret13}$|
        
\end{ffcode}

Let's denote lines \ref{ln:ret12}-\ref{ln:ret13} as Z.

As described in section 3.2.2, the program runs from top to bottom until it meets goto forward and if the jump condition is met, the abstract object g is executed again.
Similar to what was said in section 3.2.2, all objects inside the abstract object g should be copied at the jump point.

The only difference, when the jump itself frames a while loop, is that when the jump condition is met, we need to interrupt the execution of all the internal while loops inside the new while loop, which we replaced the jump with.

In order to interrupt the execution of each of the internal while loops, we can add an additional condition to each of them, whether the jump condition is met or not. This can be done with an additional flag object that will signal for each inner while loop whether the jump condition has been met.

So, all internal while loops will not be executed further if flag shows that the jump condition is met and the logic of the program is not violated. For all other objects inside our new while loop, we should apply the transformation described in section 3.2.2.

The transformation will look like this:

\begin{ffcode}

memory > flag
flag.write 0
while.
  eq. (flag 0)
  seq
    flag.write 1
    |\sgm{0}||$\label{ln:ret14}$|
      while.
        and.
          (eq. (flag 0)).not
          |\m{0}|
        |\sgm{1}|
          |\sgm{2}|
          if.
            |\m{1}|.not
            |\dlt{1}|
            flag.write 0
          if.
            (eq. (flag 0)).not
            |\sgm{3}|
            TRUE|$\label{ln:ret15}$|
        
\end{ffcode}

Let's denote lines \ref{ln:ret14}-\ref{ln:ret15} as Z'.


Let's take a look on a program with nested while-loops:
\begin{ffcode}

goto
  [g]
    while. > @
      |\m{9}|
      |\sgm{9}|
        Z
        |\sgm{4}|

\end{ffcode}

After transformation it will look like:
\begin{ffcode}

memory > flag
flag.write 0
while.
  eq. (flag 0)
  seq
    flag.write 1
    while.
      and.
        (eq. (flag 0)).not
        |\m{9}|
      |\sgm{9}|
        Z'
        if.
          (eq. (flag 0)).not
          |\sgm{4}|
          TRUE

\end{ffcode}



\subsection{Final algorithm}
To sum up, we can now fully describe the algorithm itself. The first step is to make the transformations, if necessary, described in sections 3.1.1 and 3.2.1, depending on the current transformation. The second step is going through all the jumps in the program - forward and backward, and depending on the jump we carry out the required transformations: if you need to make a replacement in the while loop, containing a jump - we use transformations 3.1.3 and 3.2.3 accordingly. If you need to make a replacement for a normal object that frames the jump - use transformations 3.1.2 and 3.2.2 respectively.



\section{Conclusion}
As a result of our research, we described an algorithm that automatically converts an input program to an equivalent analog, but without using the \emph{atom} object GOTO. We also showed that the program at the output will contain the same logic as the program at the input.

\section{Discussion}
Our software will simplify programs in EO language and make them more readable without using GOTO objects. In addition, our proof makes it possible to remove all GOTO constructs from programs.

\end{document}
\endinput

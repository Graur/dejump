\begin{eodefinition}
An \emph{object} is a collection of attributes, which are uniquely named bindings to objects.
An object is abstract if at least one of its attributes is free---isn’t bound to any object.
Objects, whose implementation is provided by the runtime are called \emph{atoms}.
One object may decorate another object by binding it to the $\varphi$ attribute of itself.
A decorator has its own attributes and bound attributes of its decoratee.
\end{eodefinition}
\begin{eodefinition}
\emph{Dataization} in $\varphi$-calculus is a process of retrieving data from an object.
The dataization of an object at the highest level of composition is similar to the execution of a program in other languages.

We also define a dataization function $\mathbb{D}(x)$, which turns an object $x$ into data.
\end{eodefinition}
\begin{eodefinition}
\emph{Forwarding goto} defines an object \ff{goto}, which encapsulate object \ff{forward} directly or through other objects:
\begin{ffcode}
goto
  [o1]
   |$\leadsto$|o1.forward o
\end{ffcode}
\end{eodefinition}
\begin{eodefinition}
\emph{Backwarding goto} defines an object \ff{goto}, which encapsulate object \ff{backward} directly or through other objects:
\begin{ffcode}
goto
  [o1]
   |$\leadsto$|o1.backward
\end{ffcode}
\end{eodefinition}

Let's consider the following EO program which demonstrates the general syntax of the language:
\begin{ffcode}
memory 0 > x|$\label{ln:def0}$|
[] > app
  [a b] > max
    goto > @
      [g]
        if. > @
          a.gt b
          g.forward a
          b
  [x] > run
    while. > @!
      and.
        x.lt 10
        (x.mod 1).eq 0
      [i]
        x.write > @
          (x.plus 1)|$\label{ln:def1}$|
\end{ffcode}
In this code snippet, \ff{goto} is an \emph{atom}-object with attribute \ff{g}, which dataizes similar to \ff{seq} object;
\ff{memory} is mutable object, which allows to reinitialize the bound object \ff{x} via \ff{.write} attribute;
\ff{eq} is an object, which checks the equality of the objects passed to it;
\ff{if} is an object, which implements behavior of branching;
\ff{seq} is an object that sequentially dataizes all objects within it;
\ff{while} is an object, implementing behavior of loops: its first bound attribute implements condition of loop dataization and second one implies "body" of loop;
\ff{backward} and \ff{forward} are objects, which implements behaviour of jumps: first of them starts dataization of object \ff{g} again, while the second one returns the object passed to it as a parameter and terminates the dataization of object \ff{goto};
\ff{and} is an objects, which implements logical operation \emph{AND}.
There are also objects \ff{or}, \ff{xor} and \ff{not} in the language, that implement the logical operators \emph{OR}, \emph{XOR} and \emph{NOT} respectively.

In terms of $\varphi$-calculus, code at \lrefs{def0}{def1} will be presented as:
\begin{equation}
\begin{split}
& app \mapsto \llbracket \br
& \quad max(a, b) \mapsto \llbracket \br
& \quad \quad \varphi \mapsto \ff{goto}(\llbracket\,g\,\mapsto\,\varnothing,\br
& \quad \quad \quad \varphi \mapsto a.\ff{gt}(b).\ff{if}(g.\ff{forward}(a), b) \,\rrbracket) \br
& \quad \rrbracket \br
& \quad run(x) \mapsto \llbracket \br
& \quad \quad \varphi\ff{!} \mapsto x.\ff{lt}(10).\ff{and}(x.\ff{mod}(1).\ff{eq}(0)) \br
& \quad \quad \quad .\ff{while}(\llbracket\,i \mapsto \varnothing, \br
& \quad \quad \quad \quad \quad \quad \quad \varphi \mapsto x.\ff{write}(x.\ff{plus}(1))\,\rrbracket) \br
& \quad \rrbracket \br
& \rrbracket \\
\end{split}
\end{equation}

\emph{Designations:}
\begin{itemize}
\item "$>$" is used to name objects, where on the left side of ">" is object itself and its name on the right side,
\item "$[]$" is used to specify object parameters,
\item "$@$" is used to denote the object being decorated,
\item "$!$" is used to declare the object as a constant,
\item "$\mapsto$" is used to defines an object, where free attributes stay in the parentheses on the left side and pairs, which represent bound attributes, stay on the right side, in double-square brackets,
\item "$\leadsto$" defines an arbitrary number of objects or chains of encapsulated objects that encapsulate the original object.
\item "$\equiv$" defines the semantic equality of objects, so result of dataization for first object is equal to result of dataization for second one:
\begin{equation}
(\forall{x, y}) (x \equiv y) \Leftrightarrow (\mathbb{D}(x) = \mathbb{D}(y)) .
\end{equation}
\end{itemize}
